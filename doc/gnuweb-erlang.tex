\section{Erlang Runtime}

\subsection{Project structure}

Erlang project contains one or more Erlang/OTP ``applications'' that composes one or more Erlang/OTP ``releases''. Those releases are standalone, self contained packages that are ready for distribution to ``target system''.\\

The typical project layout is:\\

\dirtree{%
.1 project. 
.2 app1.
.3 ebin.
.4 *.beam.
.4 app1.app.
.3 src.
.4 *.erl.
.3 tests.
.4 ....
.3 manifest.
.2 app2.
.3 ....
.2 rel1.
.3 ebin.
.4 *.beam.
.4 rel1.beam.
.4 sys.config.
.4 rel1.app.
.4 rel1.tar.gz.
.3 src.
.4 *.erl.
.4 sys.config.in.
.3 tests.
.4 ....
.3 manifest.
.2 init.
}


\begin{itemize}
\item *.erl is Erlang source code
\item *.beam is compiled Erlang code, gnuweb declares Makefile rules for Erlang source code compilation.
\item *.app is Erlang application resource file, gnuweb generates them from application manifest.
\end{itemize}



\subsection{Usage scenarios}
Config

\subsection{Macro}


\subsection{Building libraries and applications}

\subsubsection{Naming convention}

\subsubsection{Defining sources}

PHP library or application can be installed in libphpdir.

For example:
\begin{verbatim}
   libphp_SCRIPTS = libhello
\end{verbatim}

The variable libhello\_SRC is used to specify which PHP source files belongs to this library.

%If libhello_SRC is not specified, then it defaults to all php files in libhello/ directory (libhello/*.php).


\subsection{Simple Example}

\begin{thebibliography}{}
\bibitem{Erl-Rel}
http://www.erlang.org/doc/design\_principles/release\_structure.html
\end{thebibliography}


